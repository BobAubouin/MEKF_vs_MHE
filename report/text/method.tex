In this section, different methods used to estimate the unknown parameters of the PD models are detailed. $BIS_0$ can be measured at before the induction of anesthesia and $E_{max}$ is usually set equal to $BIS_0$. Thus the remaining parameters are $C_{50p}$, $C_{50r}$, and $\gamma$. in this section, $\theta = \begin{pmatrix} C_{50p} & C_{50r} & \gamma \end{pmatrix}$ is used to describe the vector of unknown parameters.\\

The Multi Extended Kalman Filter (MEKF) method select the best vector among a grid in space of the parameters. This discrete choice allow a fast convergence but less precision at the end. The Moving Horizon Estimation (MHE) method uses an extended state formulation to estimate the vector of parameters along with the state in a continuous manner. Thus the method could identify more precisely the parameters but is also more subject to noise and could be slower than MEKF.

\subsection{Multi Extended Kalman Filter}

In order to identify the PD parameters, the MEKF method uses a set of EKF, one for every realization of the vector selected within a grid in the space of the parameters. The grid is designed to reasonably represent the variability of the parameter vector. Next, a vector is chosen using a model-matching criterion. \\

EKF is a state estimation method that relies on the linearization of a non-linear model. If we consider the model given in (\ref{eq:model}) with the non-linear function $f$ parametrized by $\theta$, the estimator using the parameter vector $\theta_i$ is given by:

\begin{flalign*}
&H_i(k) = \left. \frac{\partial f(x, \theta_i)}{\partial x} \right| _{x=\hat{x}_i(k_{|k-1})} \\
&K_i(k) = P_i(k_{|k-1})H_i^\top (k)(H_i(k)P_i(k_{|k-1})H_i^\top (k) + R_2)^{-1} \\
&\hat{x}_i(k_{|k}) = \hat{x}_i(k_{|k-1}) + K_i(k)(y(k) - f(\hat{x}_i(k_{|k-1}),\theta_i )) \\
&P_i(k_{|k}) = P_i(k_{|k-1}) - K_i(k) H_i(k) P_i(k_{|k-1}) \\
&\hat{x}_i(k+1_{|k}) =  A \hat{x}_i(k_{|k}) + Bu(k) \\
&P_i(k+1_{|k}) = A P_i(k_{|k})A^\top + R_1
\end{flalign*}

Here the notation $X(k_{1|k_2})$ represents the value of variable X computed at time step $k_1$ based on the knowledge available at $k_2$. The estimated state vector is $\hat{x}$ and $P$ is the covariance matrix. $R_1$ and $R_2$ are two constant matrices used to respectively characterize the process uncertainties and the measurements noise.\\


The idea is to select the "best" observer at each step time. To do so the estimation error on the output $e_k = y_k - f(x_{k|k-1}, \theta_i)$ is used to construct a selection criterion. As in \cite{petriImprovingEstimationPerformance2022}


\subsection{Moving Horizon Estimation}
The Moving Horizon Estimator (MHE) is a dynamic system state estimation method that operates by solving an optimization problem over a moving time horizon. The optimization problem here lies in minimizing a cost function representing the discrepancy between the predicted model states and the actual measurements. The estimated states are recursively refined and updated using a combination of measured data and a mathematical model of the system. This consequently provides an accurate real-time assessment of the patient's physiological condition.\\

In the context of this paper, the MHE is used to estimate the states and the pharmacodynamics (PD) of an anesthesia model based on simulated data with known parameters. Being a model-based estimation approach, the MHE utilizes a linear PK-PD decoupled model linking the propofol and remifentanil infusion rates (up and ur) to their equivalent concentration in the effect site ($xp_4$ and $xr_4$) respectively:
\begin{flalign*}
\begin{pmatrix}\dot{x}_p \\ \dot{x}_r \end{pmatrix} =
\begin{pmatrix} A_p & 0^{4\times4 }\\0^{4\times 4} &  A_r \end{pmatrix}
\begin{pmatrix} x_p \\ x_r \end{pmatrix} + 
\begin{pmatrix} B_p & 0^{4 \times 1 } \\ 0^{4 \times 1 }  &   B_r \end{pmatrix}
\begin{pmatrix} u_p \\ u_r \end{pmatrix}
\end{flalign*}

\noindent The model can hence be simplified as follows:
\begin{flalign*}
\dot{x}=A_Tx+B_TU
\end{flalign*}

\noindent With $A_T \in R^{8\times8}$ and $B_T \in R^{8\times8}$ represent the state and the input matrices.\\

\noindent The propofol PK-PD linear model is given as:
\begin{flalign*}
\begin{pmatrix}\dot{x}_{p1}\\ \dot{x}_{p2}\\ \dot{x}_{p3}\\ \dot{x}_{p4}\end{pmatrix} = 
\begin{pmatrix}
    -a_{11p} & a_{12p} & a_{13p} & 0 \\
     a_{21p} & -a_{21p} & 0 & 0 \\  
     a_{31p} & 0 & -a_{31p} & 0 \\
     a_{41p} & 0 & 0 & -a_{41p} \\
\end{pmatrix}
\begin{pmatrix} x_{p1}\\ x_{p2}\\ x_{p3}\\ x_{p4}\\ \end{pmatrix} +
\begin{pmatrix} \frac{1}{V_{1p}}\\ 0\\ 0\\ 0 \end{pmatrix}u_p
\end{split}
\end{flalign*}

\noindent Similarly, the model can be simplified as follows:
\begin{flalign*}
\dot{x_p}=A_px_p+B_pU_p
\end{flalign*}

\noindent With $A_p \in R^{4\times4} and $B_p \in R^{4\times1}$ represent the state and the input matrices for propofol.\\

\noindent The linear PK-PD model for remifentanil follows a similar pattern.\\

The MHE offers robust state estimations in non-linear, uncertain, and constrained systems. In our case, the measured EEG signal is very noisy, so the MHE is expected to offer some benefits over the EKF that’s considered to be aggressive as it allows to fit the output with the noise.

\subsection{Metrics for the comparison}
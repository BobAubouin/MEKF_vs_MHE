In this section, different methods used to estimate the unknown parameters of the PD models are detailed. $BIS_0$ can be measured at before the induction of anesthesia and $E_{max}$ is usually set equal to $BIS_0$. Thus the remaining parameters are $C_{50p}$, $C_{50r}$, and $\gamma$. in this section, $\theta = \begin{pmatrix} C_{50p} & C_{50r} & \gamma \end{pmatrix}$ is used to describe the vector of unknown parameters.
\medskip

The Multi Extended Kalman Filter (MEKF) method select the best vector among a grid in space of the parameters. This discrete choice allow a fast convergence but less precision at the end. The Moving Horizon Estimation (MHE) method uses an extended state formulation to estimate the vector of parameters along with the state in a continuous manner. Thus the method could identify more precisely the parameters but is also more subject to noise and could be slower than MEKF.

\subsection{Multi Extended Kalman Filter}

In order to identify the PD parameters, the MEKF method uses a set of EKF, one for every realization of the vector selected within a grid in the space of the parameters. The grid is designed to reasonably represent the variability of the parameter vector. Next, a vector is chosen using a model-matching criterion. \medskip 

EKF is a state estimation method that relies on the linearization of a non-linear model. If we consider the model given in (\ref{eq:model}) with the non-linear function $f$ parametrized by $\theta$, the estimator using the parameter vector $\theta_i$ is given by:

\begin{flalign*}
&H_i(k) = \left. \frac{\partial f(x, \theta_i)}{\partial x} \right| _{x=\hat{x}_i(k_{|k-1})} \\
&K_i(k) = P_i(k_{|k-1})H_i^\top (k)(H_i(k)P_i(k_{|k-1})H_i^\top (k) + R_2)^{-1} \\
&\hat{x}_i(k_{|k}) = \hat{x}_i(k_{|k-1}) + K_i(k)(y(k) - f(\hat{x}_i(k_{|k-1}),\theta_i )) \\
&P_i(k_{|k}) = P_i(k_{|k-1}) - K_i(k) H_i(k) P_i(k_{|k-1}) \\
&\hat{x}_i(k+1_{|k}) =  A_d \hat{x}_i(k_{|k}) + B_d u(k) \\
&P_i(k+1_{|k}) = A_d P_i(k_{|k})A_d ^\top + R_1
\end{flalign*}

Here the notation $X(k_{1|k_2})$ represents the value of variable X computed at time step $k_1$ based on the knowledge available at $k_2$. The estimated state vector is $\hat{x}$ and $P$ is the covariance matrix. $A_d$ and $B_d$ are the matrix describing the discretized dynamic system (\ref{eq:model}) such that $x(k+1) = A_dx(k) + B_d u(k)$. $R_1$ and $R_2$ are two constant matrices used to respectively characterize the process uncertainties and the measurements noise.
\medskip


The idea is to select the "best" observer at each step time. To do so, the estimation error on the output $e(k) = y(k) - f(x(k|_{k-1}), \theta_i)$ is used to construct a selection criterion for each observer. As in \cite{petriImprovingEstimationPerformance2022} the dynamic of the criterion for the $i^{th}$ observer is given by:

\begin{equation}
\eta_i (k+1) = \eta_i (k) + dt \left(- \nu \eta_i(k) + \lambda_1 |e(k)|^2 + \lambda_2 |K_i(k) e(k)|^2 \right),
\label{eq:criterion_dyn}
\end{equation}

where $\lambda_1, \lambda_2$, and $\nu$ are three positive design parameters and $dt$ the sampling time of the system. The criterion depend both on the output estimation error $e(k)$ and the correction effort of the observer $K_i(k) e(k)$. Equation (\ref{eq:criterion_dyn}) can be discretize with Euler method and the following equation can be deducted:

\begin{equation}
\eta_i(k) = e^{-\nu k dt} + \sum_{j=1}^{k} e^{-\nu(k-j)dt} (\lambda_1 |e(j)|^2 + \lambda_2 |K_i(j) e(j)|^2 ).
\end{equation} 

$\eta_i$ can be seen as a cost and the idea is to select the observer with the minimal cost at each step time. The index of the actually selected observer is denoted $i^*$.

\medskip
Because this solution could produce too much switching between the observers, the parameter $\epsilon \in ]0,1]$ is introduced and the switching is done at time step $k$ only if it exists $i\neq i^*$ such that $\eta_i(k)<\epsilon \eta_{i^*}$.
\medskip

To initialized the criterion of each observer, the distribution of the parameters can be taken into account. Particularly, considering the $\mathcal{C}_i$ the subset of the parameter space such that $\theta_i$ is the closer grid point to every point in $\mathcal{C}_i$ the criterion can be initialized by:

\begin{equation}
\eta_i(0) = \frac{\alpha}{p(\theta \in \mathcal{C}_i)},
\end{equation}

where $\alpha >0$ is a design parameter.


\subsection{Moving Horizon Estimation}
The Moving Horizon Estimator (MHE) is a dynamic system state estimation method that operates by solving an optimization problem over a moving time horizon. The optimization problem here lies in minimizing a cost function representing the discrepancy between the predicted model states and the actual measurements. The estimated states are recursively refined and updated using a combination of measured data and a mathematical model of the system. This consequently provides an accurate real-time assessment of the patient's physiological condition.\\

\subsubsection{PK-PD Modeling}
In the context of this paper, the MHE is used to estimate the states and the pharmacodynamics (PD) of an anesthesia model based on simulated data with known parameters. Being a model-based estimation approach, the MHE utilizes a linear PK-PD decoupled model linking the propofol and remifentanil infusion rates (up and ur) to their equivalent concentration in the effect site ($xp_4$ and $xr_4$) respectively:
\begin{flalign*}
\begin{pmatrix}\dot{x}_p \\ \dot{x}_r \end{pmatrix} =
\begin{pmatrix} A_p & 0^{4\times4 }\\0^{4\times 4} &  A_r \end{pmatrix}
\begin{pmatrix} x_p \\ x_r \end{pmatrix} + 
\begin{pmatrix} B_p & 0^{4 \times 1 } \\ 0^{4 \times 1 }  &   B_r \end{pmatrix}
\begin{pmatrix} u_p \\ u_r \end{pmatrix}
\end{flalign*}

\noindent The model can hence be simplified as follows:
\begin{flalign*}
\dot{x}=A_Tx+B_TU
\end{flalign*}

\noindent With $A_T \in R^{8\times8}$ and $B_T \in R^{8\times8}$ represent the state and the input matrices.\\

\noindent The propofol PK-PD linear model is given as:
\begin{flalign*}
\begin{aligned}
\begin{pmatrix}\dot{x}_{p1}\\ \dot{x}_{p2}\\ \dot{x}_{p3}\\ \dot{x}_{p4}\end{pmatrix} = 
\begin{pmatrix}
    -a_{11p} & a_{12p} & a_{13p} & 0 \\
     a_{21p} & -a_{21p} & 0 & 0 \\  
     a_{31p} & 0 & -a_{31p} & 0 \\
     a_{41p} & 0 & 0 & -a_{41p} \\
\end{pmatrix}
\begin{pmatrix} x_{p1}\\ x_{p2}\\ x_{p3}\\ x_{p4}\\ \end{pmatrix} +
\begin{pmatrix} \frac{1}{V_{1p}}\\ 0\\ 0\\ 0 \end{pmatrix}u_p
\end{aligned}
\end{flalign*}

\noindent Similarly, the model can be simplified as follows:
\begin{flalign*}
\dot{x_p}=A_px_p+B_pU_p
\end{flalign*}

\noindent With $A_p \in R^{4\times4}$ and $B_p \in R^{4\times1}$ represent the state and the input matrices for propofol. The linear PK-PD model for remifentanil follows a similar pattern.\\

\subsubsection{Cost Function}
The MHE cost function is used to compute the difference between the measured and estimated outputs and controls. The parameters are penalized during the maintenance phase to eliminate the noise effect. Generally, the optimal control problem aims to minimize the cost function to obtain an accurate estimation of the states.
The cost function to be minimized at each time step $k$ can be written as follows:
\begin{equation}
\begin{split}
        J_{N_{MHE}}(\bar{\mathbf{x}}_k, \hat{\bar{\mathbf{x}}}_{k-1}, \mathbf{y}^m,\mathbf{u}^m) = \sum_{i=k-N_{MHE}}^{k}{\lVert y_i^m-h(\bar{x}_i) \rVert_Q}\\ + \sum_{i=k-N_{MHE}+1}^{k}{\lVert \bar{x}_i-\Phi(\hat{\bar{x}}_{i-1},u^m_{i-1}) \rVert_{R(k)}}
\end{split}
\end{equation}
Where $\bar{\mathbf{x}}_k$, $\hat{\bar{\mathbf{x}}}_{k-1}$, $\mathbf{y}^m$, $\mathbf{u}^m$, $Q$ and $R$, respectively represent the state vector up to time $k$ and the previously estimated state up to time $k-1$ over the estimation horizon, the output and the input measurements over the estimation horizon and the penalty matrices.\\

\subsection{Metrics for the comparison}